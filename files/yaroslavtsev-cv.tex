\def\online{0}
\def\software{0}
%%%%%%%%%%%%%%%%%%%%%%%%%%%%%%%%%%%%%%%%%%%%%%%%%%%%%%%%%%%%%%%%%%%%%%%%
%%%%%%%%%%%%%%%%%%%%%% Simple LaTeX CV Template %%%%%%%%%%%%%%%%%%%%%%%%
%%%%%%%%%%%%%%%%%%%%%%%%%%%%%%%%%%%%%%%%%%%%%%%%%%%%%%%%%%%%%%%%%%%%%%%%

%%%%%%%%%%%%%%%%%%%%%%%%%%%%%%%%%%%%%%%%%%%%%%%%%%%%%%%%%%%%%%%%%%%%%%%%
%% NOTE: If you find that it says                                     %%
%%                                                                    %%
%%                           1 of ??                                  %%
%%                                                                    %%
%% at the bottom of your first page, this means that the AUX file     %%
%% was not available when you ran LaTeX on this source. Simply RERUN  %%
%% LaTeX to get the ``??'' replaced with the number of the last page  %%
%% of the document. The AUX file will be generated on the first run   %%
%% of LaTeX and used on the second run to fill in all of the          %%
%% references.                                                        %%
%%%%%%%%%%%%%%%%%%%%%%%%%%%%%%%%%%%%%%%%%%%%%%%%%%%%%%%%%%%%%%%%%%%%%%%%

%%%%%%%%%%%%%%%%%%%%%%%%%%%% Document Setup %%%%%%%%%%%%%%%%%%%%%%%%%%%%

% Don't like 10pt? Try 11pt or 12pt
\documentclass[11pt]{article}

\interfootnotelinepenalty=10000

% This is a helpful package that puts math inside length specifications
\usepackage{calc}

% Layout: Puts the section titles on left side of page
\reversemarginpar

%
%         PAPER SIZE, PAGE NUMBER, AND DOCUMENT LAYOUT NOTES:
%
% The next \usepackage line changes the layout for CV style section
% headings as marginal notes. It also sets up the paper size as either
% letter or A4. By default, letter was used. If A4 paper is desired,
% comment out the letterpaper lines and uncomment the a4paper lines.
%
% As you can see, the margin widths and section title widths can be
% easily adjusted.
%
% ALSO: Notice that the includefoot option can be commented OUT in order
% to put the PAGE NUMBER *IN* the bottom margin. This will make the
% effective text area larger.
%
% IF YOU WISH TO REMOVE THE ``of LASTPAGE'' next to each page number,
% see the note about the +LP and -LP lines below. Comment out the +LP
% and uncomment the -LP.
%
% IF YOU WISH TO REMOVE PAGE NUMBERS, be sure that the includefoot line
% is uncommented and ALSO uncomment the \pagestyle{empty} a few lines
% below.
%

%% Use these lines for letter-sized paper

%%\usepackage[paper=letterpaper,
%            includefoot, % Uncomment to put page number above margin
%            marginparwidth=1.2in,     % Length of section titles
%           marginparsep=.05in,       % Space between titles and text
%            margin=1in,               % 1 inch margins
%           includemp]{geometry}

% Use these lines for A4-sized paper
\usepackage[paper=a4paper,
            %includefoot, % Uncomment to put page number above margin
            marginparwidth=25mm,    % Length of section titles
            marginparsep=0.7mm,       % Space between titles and text
            margin=15mm,              % 25mm margins
            includemp]{geometry}

%% More layout: Get rid of indenting throughout entire document
\setlength{\parindent}{0in}

%% This gives us fun enumeration environments. compactitem will be nice.
\usepackage{paralist}

%% Reference the last page in the page number
%
% NOTE: comment the +LP line and uncomment the -LP line to have page
%       numbers without the ``of ##'' last page reference)
%
% NOTE: uncomment the \pagestyle{empty} line to get rid of all page
%       numbers (make sure includefoot is commented out above)
%
\usepackage{fancyhdr,lastpage}
\pagestyle{fancy}
\pagestyle{empty}      % Uncomment this to get rid of page numbers
\fancyhf{}\renewcommand{\headrulewidth}{0pt}
\fancyfootoffset{\marginparsep+\marginparwidth}
\newlength{\footpageshift}
\setlength{\footpageshift}
          {0.5\textwidth+0.5\marginparsep+0.5\marginparwidth-2in}
\lfoot{\hspace{\footpageshift}%
       \parbox{4in}{\, \hfill %
                    \arabic{page} of \protect\pageref*{LastPage} % +LP
%                    \arabic{page}                               % -LP
                    \hfill \,}}

% Finally, give us PDF bookmarks
\usepackage{color,hyperref}
\definecolor{darkblue}{rgb}{0.0,0.0,0.3}
\hypersetup{colorlinks,breaklinks,
            linkcolor=blue,urlcolor=blue,
            anchorcolor=darkblue,citecolor=darkblue}

\usepackage{url}

%%%%%%%%%%%%%%%%%%%%%%%% End Document Setup %%%%%%%%%%%%%%%%%%%%%%%%%%%%


%%%%%%%%%%%%%%%%%%%%%%%%%%% Helper Commands %%%%%%%%%%%%%%%%%%%%%%%%%%%%

% The title (name) with a horizontal rule under it
%
% Usage: \makeheading{name}
%
% Place at top of document. It should be the first thing.
\newcommand{\makeheading}[1]%
        {\hspace*{-\marginparsep minus \marginparwidth}%
         \begin{minipage}[t]{\textwidth+\marginparwidth+\marginparsep}%
                {\LARGE \bfseries #1}\\[-0.15\baselineskip]%
                 \rule{\columnwidth}{1pt}%
         \end{minipage}}

% The section headings
%
% Usage: \section{section name}
%
% Follow this section IMMEDIATELY with the first line of the section
% text. Do not put whitespace in between. That is, do this:
%
%       \section{My Information}
%       Here is my information.
%
% and NOT this:
%
%       \section{My Information}
%
%       Here is my information.
%
% Otherwise the top of the section header will not line up with the top
% of the section. Of course, using a single comment character (%) on
% empty lines allows for the function of the first example with the
% readability of the second example.
\renewcommand{\section}[2]%
        {\pagebreak[2]\vspace{1.3\baselineskip}%
         \phantomsection\addcontentsline{toc}{section}{#1}%
         \hspace{0in}%
         \marginpar{
         \raggedright \scshape #1}#2}

% An itemize-style list with lots of space between items
\newenvironment{outerlist}[1][\enskip\textbullet]%
        {\begin{itemize}[#1]}{\end{itemize}%
         \vspace{-.6\baselineskip}}

% An environment IDENTICAL to outerlist that has better pre-list spacing
% when used as the first thing in a \section
\newenvironment{lonelist}[1][\enskip\textbullet]%
        {\vspace{-\baselineskip}\begin{list}{#1}{%
        \setlength{\partopsep}{0pt}%
        \setlength{\topsep}{0pt}}}
        {\end{list}\vspace{-.6\baselineskip}}

% An itemize-style list with little space between items
\newenvironment{innerlist}[1][\enskip\textbullet]%
        {\begin{compactitem}[#1]}{\end{compactitem}}

% To add some paragraph space between lines.
% This also tells LaTeX to preferably break a page on one of these gaps
% if there is a needed pagebreak nearby.
\newcommand{\blankline}{\quad\pagebreak[2]}

%%%%%%%%%%%%%%%%%%%%%%%% End Helper Commands %%%%%%%%%%%%%%%%%%%%%%%%%%%

\usepackage{verbatim}


\hyphenpenalty=10000

%%%%%%%%%%%%%%%%%%%%%%%%% Begin CV Document %%%%%%%%%%%%%%%%%%%%%%%%%%%%

\begin{document}
\sloppy

\makeheading{Grigory Yaroslavtsev, \href{http://grigory.us}{http://grigory.us}}

\section{Contact Information}
%
% NOTE: Mind where the & separators and \\ breaks are in the following
%       table.
%
% ALSO: \rcollength is the width of the right column of the table
%       (adjust it to your liking; default is 1.85in).
%
\newlength{\rcollength}\setlength{\rcollength}{2.3in}%
%
\begin{tabular}[t]{@{}p{\textwidth-\rcollength}p{\rcollength}}
121 South Main Street, \#1023 & \textit{Cell phone:} +1 (814) 713-1096 \\
Providence, RI, 02903 & \textit{E-mail:} \href{mailto:grigory@grigory.us}{grigory@grigory.us} \\
			  %& \textit{WWW:} \\
%\if\online0			  & \textit{Visa status for Fall'13:} F-1 OPT \fi
\end{tabular}

\begin{comment}

\begin{tabular}[t]{@{}p{\textwidth-\rcollength}p{\rcollength}}
357 Information Sciences and Technology Building & \textit{Cell phone:} +1 (814) 713-1096 \\
State College, PA, 16801  & \textit{E-mail:} \href{mailto:grigory@grigory.us}{grigory@grigory.us} \\
			  & \textit{WWW:} \href{http://grigory.us}{http://grigory.us}\\
\if\online=0			  & \textit{Visa status for Fall'13:} F-1 OPT \fi
\end{tabular}
\end{comment}

\begin{comment}

\section{Biographical data}
%
Citizenship: Russia

Date of birth: May 25, 1987

Place of birth: St. Petersburg, USSR

\end{comment}

\section{Research Interests}
%
Approximation and sublinear time algorithms for big data: sparsification, summarization, private data release, property testing.

\section{2013-2014}
\textbf{Brown University \href{http://icerm.brown.edu}{ICERM}}, Providence, RI.
%\begin{outerlist}
%\item[] 

\medskip
Institute Postdoctoral Fellowship in Mathematics.
%\end{outerlist}
%\section{Education}

\section{2010-2013}
\textbf{Pennsylvania State University}, State College, PA (Joined by invitation, didn't apply to any other Ph.D. programs)
%\begin{outerlist}
%\item[] 
\medskip

Ph.D., Thesis: ``Efficient Combinatorial Techniques in Sparsification, Summarization and Testing of Large Datasets.''

%    Expected graduation date: August 2013.
%	\begin{innerlist}
%	\item Advisor:\href{http://www.cse.psu.edu/~sofya}
%		{ Prof. Sofya Raskhodnikova}.
%	\end{innerlist}
%\end{outerlist}


\section{2008-2010} 
\textbf{Academic University of the Russian Academy of Sciences}, St. Petersburg, Russia
%\begin{outerlist}
%\item[]
 \medskip

 M.S. in Applied Mathematics and Physics ($1^{st}$ student in the pilot class, joint with Steklov Institute), GPA: 4.9/5.0.
%\end{outerlist}


\section{2004-2008}
\textbf{St. Petersburg State Polytechnic University}, St. Petersburg, Russia
%\begin{outerlist}
%\item[]
\medskip 

 B.S. in Physics and Technology. ($1^{st}$ result in the admission test for the department).
%\end{outerlist}

\section{Research internships}
\textbf{Microsoft Research, Redmond},  May 2013 -- August 2013.

\href{http://research.microsoft.com/en-us/groups/theory}{\textbf{Theory group}}, mentored by \href{http://www.cs.princeton.edu/~kmakaryc}{Konstantin Makarychev}.

\begin{innerlist}
\item Optimal online algorithm for online macine minimization (with N. Devanur, K. Makarychev and D. Panigrahi, in submission to ICALP'14).
\item Improved approximation algorithms for correlation clustering (with K. Makarychev, T. Schramm).
\end{innerlist}

\blankline

\textbf{Microsoft Research, SVC},  August 2012 -- October 2012.

\href{http://research.microsoft.com/en-us/labs/siliconvalley/groups/algtheory.aspx}{\textbf{Theory group}}, mentored by
\href{http://www.mit.edu/~andoni/}{Alexandr Andoni}.

\begin{innerlist}
\item Parallel algorithms for large-scale geometric problems (with A. Andoni, A. Nikolov and K. Onak, STOC'14).
\end{innerlist}

\blankline

\textbf{IBM Research, Almaden},  May 2012 -- July 2012.

\href{http://www.almaden.ibm.com/cs/disciplines/pm/}{\textbf{Theory group}}, mentored by
\href{http://www.almaden.ibm.com/cs/people/dpwoodru/}{David P. Woodruff}. 
\begin{innerlist}
\item Optimal direct-sum theorem for one-way communication complexity, showing that parallel repetition is optimal for solving multiple instances of problems, such as augmented indexing (with M. Molinaro and D. Woodruff, SODA'13). 
%This implies optimality for several sketching techniques ($n$-point Johnson-Lindenstrauss transform, multiple norm estimation, sketching matrix products, mergeable database sketches).
\item Almost optimal round vs. communication protocol for computing the intersection of distributed databases (with D. Woodruff, in submission to PODC'14, U.S. patent pending).
\end{innerlist}

\blankline

\textbf{AT\&T Labs --- Research},  May 2011 --- August 2011.

\textbf{Database theory group}, mentored by \href{http://dimacs.rutgers.edu/~graham/}{Graham Cormode}, \href{http://www2.research.att.com/~magda/}{Cecilia M. Procopiuc}, \href{http://www2.research.att.com/~divesh/}{Divesh Srivastava} and \href{http://www.research.att.com/people/Karloff_Howard_J?fbid=MB6sOrb-khf}{Howard Karloff}.
\begin{innerlist}
\item Design and implementation of efficient differentially private mechanisms for linear queries
 (with G. Cormode, M. Procopiuc and D. Srivastava, ICDE'13) %\href{http://arxiv.org/pdf/1297.6096.pdf}{ArXiv:1207.6096}.
\item Approximation algorithms for finding overlapping clusters using qualitative informaion (with H. Karloff and A. Wirth, in submission to KDD'14).
\end{innerlist}


%Hosts: Graham Cormode, Howard Karloff, Divesh Srivastava, Magda Procopiuc.

%\blankline
%
%\textbf{St. Petersburg Academic University of the Russian Academy of Sciences}
%
%Research assistant, March 2010 --- August 2010.
%

%\blankline


\newpage

\if\software1
\section{Software Engineering}
\begin{innerlist}
\item \textbf{FBReader}, Java developer, October 2007 --- April 2008. 
\begin{innerlist}
\item Worked on FBReaderJ --- the first version of FBReader for Java and Android. 
New page: \href{http://www.fbreader.org/fbreaderj}{www.fbreader.org/fbreaderj}. Old page: \href{http://old.fbreader.org/FBReaderJ/}{http://old.fbreader.org/FBReaderJ/}.
\end{innerlist}
\item \textbf{Code samples on GitHub}
\begin{innerlist}
\item Knots, Java, 2007 (Demonstrates object-oriented programming skills and familiarity with patterns of OOP, unit testing with JUnit, etc. Done as a practice project at the Academy of Modern Software Engineering): \href{https://github.com/grigory/knots}{https://github.com/grigory/knots} 
\item Shortest Paths Algorithms on Maps, C++, 2010 (Demonstrates algorithmic programming skills. Done as a project for the Microsoft Research Summer School on Algorithms and Data Stuctures): \href{https://github.com/grigory/midas-landmarks}{https://github.com/grigory/midas-landmarks}
\end{innerlist}
\end{innerlist}

\fi

\section{Achievements and awards}
\begin{innerlist}
\item \textbf{Institute Postdoctoral Fellowship in Mathematics} at Brown ICERM, 2013 --- 2014.
\item \textbf{Best Graduate Research Assistant at Computer Science and Engineering Department}, 2012.
\item \href{http://grigory.us/files/TCO10-Program.pdf}{\textbf{TopCoder Open Algorithm Competition onsite finalist (Top 24 worldwide, handle ``griffon'')}}, 2010.
\item \textbf{College of Engineering Fellowship}, 2010 --- 2013.
\item \textbf{University Graduate Fellowship}, 2010 --- 2011.
\item \textbf{Yandex personal research grant}, 2009 --- 2010.
%\item Third place in St. Petersburg Olympiad in Informatics and Programming for university students, 2009.
%\item Second diploma in International Olympiad in Informatics and Programming for university students of Russia and States of the former Soviet Union, 2009.
%%\item Third diploma in 14th All-Russian Olympiad in Informatics and Programming for university students, 2009.
%%\item Google CodeJam, local onsite finalist, 2008.
%%\item TopCoder coder of the month, April 2008.
\item Travel awards: ICDE'13, SODA'13, FOCS'12, STOC'12, STOC'11, ICALP'11, SAT'09.
\item Diploma for coaching the best team in St. Petersburg Olympiad in Informatics and Programming for high-school students, 2008.
%\item Google CodeJam Europe, 99th place. 2006.
\item $2^{nd}$ place in St. Petersburg State Polytechnic University Olympiad in Mathematics, 2005.
\item Best result in the admissions test for the Department of Physics and Technology in St. Petersburg State Polytechnic University, 2004.
\end{innerlist}

\section{Journal Papers}
\begin{lonelist}

\item \textbf{Private Analysis of Graph Structure},
with Vishesh Karwa, Sofya Raskhodnikova and Adam Smith.

ACM Transactions on Database Systems, to appear.

\item \textbf{Steiner Transitive-Closure Spanners of Low-Dimensional Posets},
with Piotr Berman, Arnab Bhattacharyya, Elena Grigorescu, Sofya Raskhodnikova and David Woodruff.

Combinatorica, to appear.

\item \textbf{Approximation Algorithms for Spanner Problems and Directed Steiner Forest}, with Piotr Berman, Arnab Bhattacharyya, Konstantin Makarychev and Sofya Raskhodnikova.

Information and Computation, special issue for ICALP'11. Volume 222, 2013, pp. 93-107.

\item \href{http://grigory.us/files/publications/2010_upper_bounds_symmetric_ipl.pdf}{\textbf{New upper bounds on the Boolean Circuit Complexity of Symmetric Functions}}, with Eugeny Demenkov, Arist Kojevnikov and Alexander Kulikov.

Information Processing Letters, 110, pp. 264-267, Elsevier, 2010.

\end{lonelist}

\section{Submitted}
\begin{lonelist}
\item \textbf{Online Algorithms for Machine Minimization}, with Nikhil Devanur, Konstantin Makarychev and Debmalya Panigrahi.

Tech report: \href{http://arxiv.org/abs/1403.0486}{ArXiv:1403.0486}.

\item \textbf{The Information Complexity of Certifying Equality and Finding the Intersection}, with Joshua Brody, Amit Chakrabarti, Ranganath Kondapally and David Woodruff.

In submission. 
\if\online0 
Draft: \href{https://www.dropbox.com/s/kbu6a1202749xg4/bckwy.pdf}{PDF}.
\else Available upon request.\fi

\item \textbf{Correlation Clustering with Overlaps}, with Howard Karloff, Jessica McClintock, Charalampos Tsourakakis and Anthony Wirth.

\if\online0 
Draft: \href{https://www.dropbox.com/s/7erlyx6zpcdrw7o/kwy.pdf}{PDF}.
\else Available upon request.\fi

\end{lonelist}

\section{Conference Papers}
\begin{lonelist}

\item \textbf{Beyond Set Disjointness: The Communication Complexity of Finding the Intersection}, with Joshua Brody, Amit Chakrabarti, Ranganath Kondapally and David Woodruff.

PODC 2014 (33rd Annual ACM SIGACT-SIGOPS Symposium on Principles of Distributed Computing).

\item \textbf{Parallel Algorithms for Geometric Graph Problems}, with Alexandr Andoni, Krzysztof Onak and Aleksandar Nikolov.

STOC 2014 (46th ACM Symposium on the Theory of Computing). 
\if\online0 
Draft: \href{https://www.dropbox.com/s/tnckb9xe3xkxu7l/anoy.pdf}{PDF}.
\else Available upon request.\fi
Tech report: \href{http://arxiv.org/abs/1401.0042}{ArXiv:1401.0042}.

\item \textbf{$L_p$-testing}, with Piotr Berman and Sofya Raskhodnikova.

STOC 2014 (46th ACM Symposium on the Theory of Computing). 
\if\online0 
Draft: \href{https://www.dropbox.com/s/q3li9u4hpob4y9z/bry.pdf}{PDF}.
\else Available upon request.\fi

\item \textbf{Lower Bounds for Testing Properties of Functions over Hypergrid Domains}, with Eric Blais and Sofya Raskhodnikova.

CCC 2014 (29th IEEE Conference on Computational Complexity). 
\if\online0 
Draft: \href{https://www.dropbox.com/s/lxbiymajfmocd66/bry2.pdf}{PDF}. 
\else Available upon request. \fi
 Tech report: 
\href{http://eccc.hpi-web.de/report/2013/036/}{ECCC TR13-036}.


\item \footnote{This is the only paper with non-alphabetical ordering of authors}\textbf{Accurate and Efficient Private Release of Datacubes and Contingency Tables}. 
Grigory Yaroslavtsev, Graham Cormode, Cecilia M. Procopiuc and Divesh Srivastava.

ICDE 2013 (29th IEEE International Conference on Data Engineering). Available as \href{http://arxiv.org/pdf/1207.6096.pdf}{ArXiv:1207.6096}.

\item \textbf{Beating the Direct Sum Theorem in Communication Complexity with Implications for Sketching}, with Marco Molinaro and David Woodruff.

SODA 2013 (24th Annual ACM-SIAM Symposium on Discrete Algorithms).

\item \textbf{Learning Pseudo-Boolean k-DNF and Submodular Functions}, with Sofya Raskhodnikova.

SODA 2013 (24th Annual ACM-SIAM Symposium on Discrete Algorithms).

\item \textbf{Primal-dual algorithms for Node-Weighted Network Design in Planar Graphs}, with Piotr Berman.

APPROX 2012 (15th International Workshop on Approximation Algorithms for Combinatorial Optimization Problems).

\item \href{http://grigory.us/files/PrivateGraphStructure-VLDB11.pdf}{
    \textbf{Private Analysis of Graph Structure}},
with Vishesh Karwa, Sofya Raskhodnikova and Adam Smith.

VLDB 2011 (37th International Conference on Very Large Data Bases), Research track.

\item \href{http://grigory.us/files/publications/directed-spanners-journal-submitted.pdf}
{\textbf{Improved Approximation for the Directed Spanner Problem}}, with Piotr Berman, Arnab Bhattacharyya, Konstantin Makarychev and Sofya Raskhodnikova.

ICALP 2011 (38th International Colloquium on Automata, Languages and Programming), Track A. 

\textbf{Runner-up for the Best Paper Award, invited to a special issue of a journal "Information and Computation"}.

\item \href{http://grigory.us/files/steiner-ICALP11.pdf}{\textbf{Steiner Transitive-Closure Spanners of Low-Dimensional Posets}},
with Piotr Berman, Arnab Bhattacharrya, Elena Grigorescu, Sofya Raskhodnikova and David Woodruff.

ICALP 2011 (38th International Colloquium on Automata, Languages and Programming), Track A.

\item \href{http://grigory.us/files/publications/0903_SAT2009_Efficient_Boolean_Circuits.pdf}{\textbf{Finding Efficient Circuits using SAT-solvers}}, with
Arist Kojevnikov and Alexander Kulikov.

SAT 2009 (12th International Conference on Theory and Applications of Satisfiability Testing).

\end{lonelist}

\section{Research visits}
\begin{innerlist}
\item Microsoft Research, Redmond. 01/08/14--01/12/14. (Host: Konstantin Makarychev)
\item IBM T.J. Watson Research Center, Yorktown Heights, NY. 04/19/11--04/21/11, 11/13/12--11/15/12. (Hosts: Konstantin Makarychev, Vishwanath Nagarajan)
\item AT\&T Labs -- Research, Florham Park, NJ. 11/18/11--11/25/11. (Host: Howard Karloff)
\item Weizmann Institute of Science, Rehovot, Israel. 12/27/12--01/04/13. (Host: Robert Krauthgamer)
\item University of Melbourne, Australia. 04/12/13--04/20/13. (Host: Anthony Wirth)
\item Aarhus University, Denmark. 05/17/13--05/25/13. (Host: Joshua Brody)
\end{innerlist}

\section{Long talks + posters}


\begin{innerlist}
\item ``The Big Data Theory'' and Randomized Algorithms
\begin{innerlist}
\item Georgia Tech, Atlanta, GA. March 2014.
\end{innerlist}
\item Approximating Graph Problems: The Old and The New
\begin{innerlist}
\item Yahoo! Research, NYC. February 2014.
\item MIT, Boston, MA. Algorithms and Complexity Seminar. February 2014.
\item Toyota Technological Institute, Chicago IL. February 2014.
\end{innerlist}
\item Testing Properties Under $L_p$ Distances
\begin{innerlist}
\item Microsoft Research, Redmond. Theory Lunch. January 2014.
\item Harvard University, Boston MA. Theory Seminar. November 2013.
\item Brown University, Providence RI. Theory Seminar. November 2013.
\item IBM Almaden Research Center, San Jose, CA. October 2013.
\end{innerlist}

\item Property Testing and Communication Complexity
\begin{innerlist}
\item MIT, Boston, MA. Algorithms and Complexity seminar. September 2013.
\end{innerlist}

\item Accurate and Efficient Private Release of Data Cubes and Contingency Tables
\begin{innerlist}
\item Cornell University, CDI project meeting. May 2013.
\end{innerlist}

\item Beating the Direct Sum in Communication Complexity with Implications for Sketching.
\begin{innerlist}
\item Aarhus University, Denmark. Theory seminar. May 2013.
\item MIT, Boston, MA. Algorithms and Complexity seminar. December 2012.
\item Princeton University, Princeton, NJ. Theory lunch. November 2012.
\item FOCS 2012, Rutgers University, NJ. Poster session. October 2012.
\end{innerlist}

\item Overlapping Clustering with Qualitative Information
\begin{innerlist}
\item FOCS 2012, Rutgers University, NJ. Poster session. October 2012.
\end{innerlist}

\item Parallel Algorithms for Geometric Problems
\begin{innerlist}
\item Sandia Labs, Livermore, CA. March 2014.
\item Stanford University, Stanford, CA. March 2014.
\item Microsoft Research SVC, Mountain View, CA. Group meeting. October 2012.
\item FOCS 2012, Rutgers University, NJ. Poster session. October 2012.
\end{innerlist}

\item Learning and Testing Submodular Functions.
\begin{innerlist}
  \item Microsoft Research, Redmond. Theory seminar. June 2013.
  \item University Of Melbourne, Theory seminar, April 2013.
  \item UCLA, Los Angeles, LA. Theory seminar. February 2013.
  \item Weizmann Institute of Science, Rehovot, Israel. Theory seminar. December 2012.  
  \item New York Computer Science and Economics Day (NYCE), New York, NY. Poster session. December 2012.  
  \item Harvard University, Boston, MA. Theory seminar. December 2012.
  \item Carnegie-Mellon University, Pittsburgh, PA. Theory lunch, December 2012.
  \item Carnegie-Mellon University, Pittsburgh, PA. OR seminar. December 2012.
  \item IBM T.J. Watson Research Center, Yorktown Heights, NY. IP for lunch. November 2012.
  \item Columbia University, New York, NY. October 2012.
  \item FOCS 2012, Rutgers University, NJ. Poster session. October 2012.
  \item Microsoft Research SVC, Mountain View, CA. Theory seminar. October 2012.
  \item EPFL, Lausanne, Switzerland. Algorithmic Frontiers Workshop, poster session. June 2012.
  \item STOC 2012, New York, NY. Poster session. May 2012.
  \item IBM Almaden Research Center, San Jose, CA. Theory seminar. May 2012.
\end{innerlist}

\item \href{http://grigory.us/files/talks/AdvancesInDirectedSpanners.pdf}{Advances in Directed Spanners}.
\begin{innerlist}
\item University of Sydney, Theory seminar, April 2013.
\item Carnegie-Mellon University, Theory Lunch, November 2011.
\item University of Maryland, Capital Area Theory Seminar, November 2011.
\end{innerlist}

\item \href{http://grigory.us/files/talks/KRSY-VLDB11.pptx}{Private Analysis of Graph Structure}
\begin{innerlist}
  \item EPFL, Lausanne, Switzerland. Algorithmic Frontiers Workshop, poster session. June 2012.
  \item AT\&T Labs --- Research, Florham Park, NJ. August 2011.
\end{innerlist}

\item Introduction to Property Testing
\begin{innerlist}
\item St. Petersburg Department of V.A. Steklov Institute of Mathematics of the Russian Academy of Sciences. December 2012.
\item St. Petersburg Institute of Fine Mechanics and Optics. Theory seminar. December 2012.
\end{innerlist}

\item \href{http://grigory.us/talsk/DPIntro.pptx}{Introduction to Differential Privacy} (based on slides by Adam D. Smith).
\begin{innerlist}
\item St. Petersburg Department of Steklov Institute of Mathematics, Computer Science club, May 2011.
\end{innerlist}

\item Property Testing and Communication Complexity
\begin{innerlist}
\item Moscow State University. Kolmogorov seminar. May 2011.
\end{innerlist}


\item \href{http://grigory.us/files/talks/20110704_Directed_Spanners_ICALP11.pptx}{Improved Approximation for the Directed Spanner Problem}
\begin{innerlist}
  \item AT\&T Labs --- Research, Florham Park, NJ. Mathematics Research Colloquium and Informal Seminar. June 2011.
  \item Princeton, NJ, June 2011. Workshop on Approximation Algorithms, Open Problem Session: \href{http://grigory.us/files/directed-spanners-open-problems.pptx}{[Slides]}, \href{http://grigory.us/files/grigory.pdf}{[Notes by \href{http://www.cs.umd.edu/~rchitnis/}{Rajesh Chitnis}]}.
  \item STOC 2011, San Jose, CA. Poster session. June 2011.
  \item Moscow State University. Combinatorial optimization seminar. May 2011.
  \item IBM T.J. Watson Research Center, Yorktown Heights, NY. IP for lunch. April 2011.
  \item St. Petersburg Institute of Fine Mechanics and Optics. Theory seminar. December 2010.

\end{innerlist}
\item \href{http://grigory.us/files/talks/penn10.pdf}{Linear bounds on circuit complexity and feebly one-way permutations}
    \begin{innerlist}
      \item Pennsylvania State University, State College, PA. Theory seminar. April 2010.
    \end{innerlist}
\end{innerlist}
\blankline

%Conference talks:
%
%\begin{innerlist}
%\item \href{http://grigory.us/files/talks/KRSY-VLDB11.pptx}{Private Analysis of Graph Structure}
%    \begin{innerlist}
%        \item Very Large Databases (VLDB 2011, Research track), August 2011.
%    \end{innerlist}
%\item \href{http://grigory.us/files/talks/20110704_Directed_Spanners_ICALP11.pptx}{Improved Approximation for the Directed Spanner Problem}
%    \begin{innerlist}
%    \item International Colloquium on Automata, Languages and Programming (ICALP 2011, Track A), July 2011.
%    \end{innerlist}
%\item \href{http://grigory.us/files/talks/20110708_Steiner-TC-Spanners.ppsx}{Steiner Transitive-Closure Spanners of Low-Dimensional Posets}
%    \begin{innerlist}
%    \item International Colloquium on Automata, Languages and Programming (ICALP 2011, Track A), July 2011.
%    \end{innerlist}
%\end{innerlist}

%
%\item \href{http://web.mit.edu/matulef/www/papers/PTviaCC-full.pdf}{Property Testing Lower Bounds via Communication Complexity} (based on a paper by Blais, Brody and Matulef).
%\begin{innerlist}
%  \item Moscow State University. Kolmogorov seminar. May 2011.
%\end{innerlist}
%
%\item \href{http://grigory.us/talks/20101224_Deterministic_extraction.pdf}{Two-Party Differential Privacy and Deterministic Extraction from Santha-Vazirani Sources} (based on a paper by McGregor, Mironov, Pitassi, Reingold, Talwar and Vadhan).
%
%        \begin{innerlist}
%            \item Pennsylvania State University, Theory seminar, November 2010.
%            \item St. Petersburg Department of Steklov Institute of Mathematics, Computer Science club, December 2010.
%            \item St. Petersburg Department of Steklov Institute of Mathematics, Seminar on discrete mathematics, December 2010.
%        \end{innerlist}

\begin{comment}
\item \href{http://grigory.us/talks/20100601_MSc_Thesis.pdf}{Finding efficient boolean circuits using SAT-solvers}, St. Petersburg Academic University, June, 2010.
\item \href{http://grigory.us/talks/20100405_Linear_bounds.pdf}{Linear bounds on circuit complexity and feebly one-way permutations}, PSU, April 2010.
\item \href{http://logic.pdmi.ras.ru/aptu/?q=node/16}{Satisfiability Modulo Theories}, St. Petersburg Academic University theory seminar,  November, 2009
\item \href{http://logic.pdmi.ras.ru/aptu/?q=node/13}{SAT-solvers: Theory and practice}, St. Petersburg Academic University theory seminar, November, 2009
\item \href{http://grigory.us/talks/20100517_Applying_Practice_to_Theory.pdf}{Applying practice to theory}, St. Petersburg Department of Steklov Institute of Mathematics Computer Science club, May 2009
\item \href{http://grigory.us/talks/0904JASS/Yaroslavtsev_Lower_bounds_using_cc.pdf}{Lower bounds using communication complexity}, Joint Advanced Student School'09, April 2009
\item Circuit complexity of MOD-functions, 51-st Scientific Conference of Moscow Instituite of Physics and Technology, December 2008.
\item \href{http://grigory.us/talks/DMSeminar.html}{Communication complexity and circuit depth of boolean functions}, St. Petersburg Department of Steklov Institute of Mathematics seminar on discrete mathematics, December 2008
\item \href{http://grigory.us/talks/chebyshev.pdf}{Plane trees and generalised Chebyshev polynomials}, Joint Advanced Student School'08, March 2008
\end{comment}




\section{Patents}

\begin{innerlist}
\item ``A Communication and Message-Efficient Protocol for Computing the Intersection Between Different Sets of Data'', with David P. Woodruff. U.S. patent pending. IBM Almaden Research Center, San Jose, CA.
\end{innerlist}


\section{Service}
Organized \href{http://grigory.us/theory-seminar-brown-fall13.html}{Theory Seminar} at Brown CS Department and Brown ICERM (2013 -- 2014).

\blankline

PC member for ``Graph Theory and Applications'' CSEDays 2012, Ekaterinburg, Russia.

\blankline


Reviewer for:

\begin{innerlist}
\item Journals: SIAM Journal on Computing (SICOMP), SIAM Journal on Discrete Mathematics (SIDMA), Information and Computation (I\&C), IEEE Transactions on Knowledge and Data Engineering (TKDE), Theory of Computing (ToC).
\item Conferences: FOCS'14, ICALP'14, FOCS'13, MFCS'13, ICALP'13, SODA'13, APPROX'12, FOCS'12, COCOA'12, SWAT'12, SODA'12, VLDB'12, WADS'11, MFCS'10, SAT'10.
\end{innerlist}



%Member of:
%\begin{innerlist}
%  \item ACM Special Interest Group on Algorithms and Computation Theory (ACM SIGACT)
%  \item European Association for Theoretical Computer Science (EATCS)
%\end{innerlist}
\blankline



\section{Teaching}
15-hour crash course ``\href{http://grigory.us/big-data.html}{Sublinear Algorithms for Big Data}'':
\begin{innerlist}
\item Brown University, ICERM. Spring 2014.
\item University of Buenos Aires, Argentina. July -- August 2014.
\end{innerlist}

\blankline

Organized a theory reading group at Penn State (running meetings / selection of material):
\begin{innerlist}
 \item Spring 2013: ``Computer Science for the Information Age'', based on an eponimous book by John Hopcroft and Ravi Kannan.
 \item Fall 2011: Selected lectures from classes \href{http://www.cs.cmu.edu/~odonnell/boolean-analysis/}{Analysis of Boolean Functions} by Ryan O'Donnell and \href{http://www.cs.washington.edu/education/courses/cse533/05au/}{The PCP Theorem and Hardness of Approximation} by Venkatesan Guruswami and Ryan O'Donnell at CMU.
  \item Spring 2011: \href{http://www.cs.princeton.edu/courses/archive/fall02/cs597D/}{``A Theorist's Toolkit''}, based on notes for a class taught by Sanjeev Arora at Princeton.
\end{innerlist}

\blankline

Extracurricular education for high-school students:

\begin{innerlist}
\item Prepared training contests for the United States team in International Olympiad in Informatics 2011.

\item Co-founder and coordinator of St. Petersburg network of extracurricular education in informatics for high-school students (\href{http://spbtc.ru}{http://spbtc.ru}) (2009-2010).

\item Judge for Baltic Science and Engineering Contest (Intel ISEF semifinals), 2010.
\
\end{innerlist}


\begin{comment}

\section{Teaching}
Course teacher in Olympiad Programming in Physical-Technical Lyceum, St. Petersburg (2005 - 2010).

\blankline

Course teacher in Summer Informatics School for winners of the All-Russian Olympiad in Informatics and Programming (2008), visiting lecturer (2009-2010).

\blankline

Course teacher in St. Petersburg Training Camp for winners of St. Petersburg Olympiad in Informatics for School Students (2008).

\end{comment}

\begin{comment}

\blankline

Teacher in Summer Camp for winners of the St. Petersburg Olympiad in Physics (2006, 2007).

\end{comment}

\section{Professional Membership}
\blankline
ACM SIGACT, IEEE

%\section{Summer schools, workshops, conferences}
%\begin{outerlist}
%
%\item Workshops:
%\begin{innerlist}
%  \item Princeton Workshop on Approximation Algorithms, Princeton, NJ, June 2011.
%  Presentation at the Open Problem Session: \href{http://grigory.us/files/directed-spanners-open-problems.pptx}{[Slides]}, \href{http://grigory.us/files/grigory.pdf}{[Notes by \href{http://www.cs.umd.edu/~rchitnis/}{Rajesh Chitnis}]}.
%\end{innerlist}
%
%\item Summer schools:
%\begin{innerlist}
%\item \href{http://logic.pdmi.ras.ru/midas}{Microsoft Research Data Structures and Algorithms School (MIDAS)}, St. Petersburg, August 2010.
%Lecturers: Andrew Goldberg, David S. Johnson, Robert E. Tarjan, Giuseppe Italiano, Renato Werneck.
%\item \href{http://www.mpi-inf.mpg.de/VTSA09}{Verification Technology, Systems and Applications}, INRIA, Nancy, October 2009
%Lecturers: Daniel Le Berre, Leonardo de Moura, Stepan Schulz, Benjamin Werner, Patricia Bouyer.
%\item \href{http://www.illc.uva.nl/LogicList/newsitem.php?id=2967}{Fall School on Logic and Complexity}, Charles University, Prague, September 2009
%Lecturers: Samuel R. Buss, Ran Raz, Pavel Pudlak, Neil Thapen.
%\item \href{http://logic.pdmi.ras.ru/ssct09}{NoNA Summer School on Complexity Theory}, St. Petersburg, August 2009
%Lecturers: Lance Fortnow, Heribert Vollmer, Nicola Galesi, Uri Zwick.
%\item Microsoft Research Summer School on High-Performance Computing, Moscow State University, July 2009
%\item Joint Advanced Student School (Proof complexity track, Trees and their applications track), St. Petersburg, 2009-2010
%%\item Petrozavodsk training camp for winners of collegiate competitions in programming, Petrozavodsk, 2007 - 2009.
%\end{innerlist}
%
%\item Conferences: FOCS'10, SODA'11, STOC'11, ICALP'11, Random/Approx'11, ITCS'12.
%Supported by NSF, IEEE TCMF, EATCS and ACM SIGACT travel grants.
%\end{outerlist}

\section{Technical skills}
C/C++, STL, Java, Windows/Linux, \LaTeX, CPlex/Gurobi/AMPL.

%\blankline
%
%\section{References}
%\begin{lonelist}
%  \item Sofya Raskhodnikova (advisor), Assistant Professor, Pennsylvania State University. \href{mailto:sofya@cse.psu.edu}{sofya@cse.psu.edu}
%  \item Adam D. Smith, Associate Professor, Pennsylvania State University. \href{mailto:asmith@cse.psu.edu}{asmith@cse.psu.edu}
%  \item Piotr Berman, Associate Professor, Pennsylvania State University. \href{mailto:berman@cse.psu.edu}{berman@cse.psu.edu}
%\end{lonelist}

\end{document}

%%%%%%%%%%%%%%%%%%%%%%%%%% End CV Document %%%%%%%%%%%%%%%%%%%%%%%%%%%%%
