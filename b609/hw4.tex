%
% This is a borrowed LaTeX template file for lecture notes for CS267,
% Applications of Parallel Computing, UCBerkeley EECS Department and
%  CMU's 10725 Fall 2012 Optimization course
% taught by Geoff Gordon and Ryan Tibshirani.  
%Now it is used for CSCI-B609 Foundations of Data Science 
%course at Indiana  University. When preparing 
% your homework for this class, please use this template.
%
% To familiarize yourself with this template, the body contains
% some examples of its use.  Look them over.  Then you can
% run LaTeX on this file.  After you have LaTeXed this file then
% you can look over the result either by printing it out with
% dvips or using xdvi. "pdflatex template.tex" should also work.
%

\documentclass[twoside]{article}
\setlength{\oddsidemargin}{0.25 in}
\setlength{\evensidemargin}{-0.25 in}
\setlength{\topmargin}{-0.6 in}
\setlength{\textwidth}{6.5 in}
\setlength{\textheight}{8.5 in}
\setlength{\headsep}{0.75 in}
\setlength{\parindent}{0 in}
\setlength{\parskip}{0.1 in}

%
% ADD PACKAGES here:
%

\usepackage{amsmath,amsfonts,graphicx}
\usepackage{hyperref}

%
% The following commands set up the hwnum (homework number)
% counter and make various numbering schemes work relative
% to the homework number.
%
\newcounter{hwnum}
\renewcommand{\thepage}{\thehwnum-\arabic{page}}
\renewcommand{\thesection}{\thehwnum.\arabic{section}}
\renewcommand{\theequation}{\thehwnum.\arabic{equation}}
\renewcommand{\thefigure}{\thehwnum.\arabic{figure}}
\renewcommand{\thetable}{\thehwnum.\arabic{table}}

%
% The following macro is used to generate the header.
%
\newcommand{\hw}[4]{
   \pagestyle{myheadings}
   \thispagestyle{plain}
   \newpage
   \setcounter{hwnum}{#1}
   \setcounter{page}{1}
   \noindent
   \begin{center}
   \framebox{
      \vbox{\vspace{2mm}
    \hbox to 6.28in { {\bf CSCI B609 -- Foundations of Data Science
	\hfill Fall 2016} }
       \vspace{4mm}
       \hbox to 6.28in { {\Large \hfill Homework #1: #2  \hfill} }
       \vspace{2mm}
       \hbox to 6.28in { {\it Name: #3 \hfill #4} }
      \vspace{2mm}}
   }
   \end{center}


}
%
% Convention for citations is authors' initials followed by the year.
% For example, to cite a paper by Leighton and Maggs you would type
% \cite{LM89}, and to cite a paper by Strassen you would type \cite{S69}.
% (To avoid bibliography problems, for now we redefine the \cite command.)
% Also commands that create a suitable format for the reference list.
\renewcommand{\cite}[1]{[#1]}
\def\beginrefs{\begin{list}%
        {[\arabic{equation}]}{\usecounter{equation}
         \setlength{\leftmargin}{2.0truecm}\setlength{\labelsep}{0.4truecm}%
         \setlength{\labelwidth}{1.6truecm}}}
\def\endrefs{\end{list}}
\def\bibentry#1{\item[\hbox{[#1]}]}

%Use this command for a figure; it puts a figure in wherever you want it.
%usage: \fig{NUMBER}{SPACE-IN-INCHES}{CAPTION}
\newcommand{\fig}[3]{
			\vspace{#2}
			\begin{center}
			Figure \thehwnum.#1:~#3
			\end{center}
	}
% Use these for theorems, lemmas, proofs, etc.
\newtheorem{theorem}{Theorem}[hwnum]
\newtheorem{lemma}[theorem]{Lemma}
\newtheorem{proposition}[theorem]{Proposition}
\newtheorem{claim}[theorem]{Claim}
\newtheorem{corollary}[theorem]{Corollary}
\newtheorem{definition}[theorem]{Definition}
\newenvironment{proof}{{\bf Proof:}}{\hfill\rule{2mm}{2mm}}
\newtheorem{problem}[theorem]{Problem}


% **** IF YOU WANT TO DEFINE ADDITIONAL MACROS FOR YOURSELF, PUT THEM HERE:

\newcommand\E{\mathbb{E}}

\begin{document}

\hw{4}{November 23}{YOUR NAME HERE}{Due: Monday, December 05, 11:59pm EST}

\begin{problem}[Sparse recovery]
	In this problem all vectors are in $\mathbb R^n$.
	Recall that the sparse recovery error is defined as:
	$$Err^k(f) = \min_{g: \|g\|_0 = k} \|f-g\|_1.$$
	Give a formal argument that shows that $Err^k(f) = \sum_{i \in S} |f_i|$ where $S$ is the set of indices of $k$ largest (by absolute value) entries of $f$.
\end{problem} 


\begin{problem}[Dyadic intervals]
Let $n$ be a power of two. Consider the following family of partitions of the interval $1, \dots, n$ into intervals:
\begin{align*}
&I_0 = \{\{1\}, \{2\} \dots, \{n\}\}\} \\
&I_1 = \{\{1,2\}, \{3,4\}, \{5,6\}, \dots, \{n - 1, n\}\} \\
&I_2 = \{\{1,2,3,4\}, \{5,6,7,8\}, \dots, {n -3, n -2, n-1, n}\} \\
\dots \\
&I_{\log n} = \{\{1, \dots, n\}\},
\end{align*}
where the partition $I_k$ consists of intervals of length $2^k$.
Show that any subinterval $i, \dots, j$ where $i \le j$ can be represented as a disjoint union of at most $2 \log n$ intervals from the above family.
\end{problem}


\begin{problem}[Generating uniform distribution]
Given a stream of numbers $a_1, \dots, a_n$ where each number is an integer between $1$ and $m$ design an algorithm that scans the stream and at every point during the scan maintains a uniformly at random chosen sample of $k$ numbers from the stream. Your algorithm should use space $O(k \log m)$.
\end{problem}

\begin{problem}[Bipartiteness via connectivity]
Consider the following reduction: given a connected undirected graph $G(V,E)$ construct a new graph $G'(V_1 \cup V_2, E')$ where $V_1$ and $V_2$ are copies of $V$ and for each edge $(u,v) \in E$ we create two edges $(u_1, v_2)$ and $(u_2, v_1)$ where $u_i$ and $v_i$ are copies of $u$ and $v$ in $V_i$.
Prove the following two statements:
\begin{enumerate}
\item If $G$ is bipartite then the number of connected components in $G'$ equals $2$.
\item If $G$ is non-bipartite then $G'$ is connected.
\end{enumerate}
\end{problem}



\end{document}





