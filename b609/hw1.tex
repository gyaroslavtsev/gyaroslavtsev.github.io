%
% This is a borrowed LaTeX template file for lecture notes for CS267,
% Applications of Parallel Computing, UCBerkeley EECS Department and
%  CMU's 10725 Fall 2012 Optimization course
% taught by Geoff Gordon and Ryan Tibshirani.  
%Now it is used for CSCI-B609 Foundations of Data Science 
%course at Indiana  University. When preparing 
% your homework for this class, please use this template.
%
% To familiarize yourself with this template, the body contains
% some examples of its use.  Look them over.  Then you can
% run LaTeX on this file.  After you have LaTeXed this file then
% you can look over the result either by printing it out with
% dvips or using xdvi. "pdflatex template.tex" should also work.
%

\documentclass[twoside]{article}
\setlength{\oddsidemargin}{0.25 in}
\setlength{\evensidemargin}{-0.25 in}
\setlength{\topmargin}{-0.6 in}
\setlength{\textwidth}{6.5 in}
\setlength{\textheight}{8.5 in}
\setlength{\headsep}{0.75 in}
\setlength{\parindent}{0 in}
\setlength{\parskip}{0.1 in}

%
% ADD PACKAGES here:
%

\usepackage{amsmath,amsfonts,graphicx}

%
% The following commands set up the hwnum (homework number)
% counter and make various numbering schemes work relative
% to the homework number.
%
\newcounter{hwnum}
\renewcommand{\thepage}{\thehwnum-\arabic{page}}
\renewcommand{\thesection}{\thehwnum.\arabic{section}}
\renewcommand{\theequation}{\thehwnum.\arabic{equation}}
\renewcommand{\thefigure}{\thehwnum.\arabic{figure}}
\renewcommand{\thetable}{\thehwnum.\arabic{table}}

%
% The following macro is used to generate the header.
%
\newcommand{\hw}[4]{
   \pagestyle{myheadings}
   \thispagestyle{plain}
   \newpage
   \setcounter{hwnum}{#1}
   \setcounter{page}{1}
   \noindent
   \begin{center}
   \framebox{
      \vbox{\vspace{2mm}
    \hbox to 6.28in { {\bf CSCI B609 -- Foundations of Data Science
	\hfill Fall 2016} }
       \vspace{4mm}
       \hbox to 6.28in { {\Large \hfill Homework #1: #2  \hfill} }
       \vspace{2mm}
       \hbox to 6.28in { {\it Name: #3 \hfill #4} }
      \vspace{2mm}}
   }
   \end{center}


}
%
% Convention for citations is authors' initials followed by the year.
% For example, to cite a paper by Leighton and Maggs you would type
% \cite{LM89}, and to cite a paper by Strassen you would type \cite{S69}.
% (To avoid bibliography problems, for now we redefine the \cite command.)
% Also commands that create a suitable format for the reference list.
\renewcommand{\cite}[1]{[#1]}
\def\beginrefs{\begin{list}%
        {[\arabic{equation}]}{\usecounter{equation}
         \setlength{\leftmargin}{2.0truecm}\setlength{\labelsep}{0.4truecm}%
         \setlength{\labelwidth}{1.6truecm}}}
\def\endrefs{\end{list}}
\def\bibentry#1{\item[\hbox{[#1]}]}

%Use this command for a figure; it puts a figure in wherever you want it.
%usage: \fig{NUMBER}{SPACE-IN-INCHES}{CAPTION}
\newcommand{\fig}[3]{
			\vspace{#2}
			\begin{center}
			Figure \thehwnum.#1:~#3
			\end{center}
	}
% Use these for theorems, lemmas, proofs, etc.
\newtheorem{theorem}{Theorem}[hwnum]
\newtheorem{lemma}[theorem]{Lemma}
\newtheorem{proposition}[theorem]{Proposition}
\newtheorem{claim}[theorem]{Claim}
\newtheorem{corollary}[theorem]{Corollary}
\newtheorem{definition}[theorem]{Definition}
\newenvironment{proof}{{\bf Proof:}}{\hfill\rule{2mm}{2mm}}
\newtheorem{problem}[theorem]{Problem}


% **** IF YOU WANT TO DEFINE ADDITIONAL MACROS FOR YOURSELF, PUT THEM HERE:

\newcommand\E{\mathbb{E}}

\begin{document}

\hw{1}{September 12}{YOUR NAME HERE}{Due: September 25, 11:59pm EST}

\begin{problem}[Exercise 2.4]
	The Chebyshev inequality states that $Pr[|X - \mathbb E[X]| \ge k \sqrt{Var[X]}] \le \frac{1}{k^2}$ for $k \ge 1$.
	Give a probability distribution and a value $k$ for which:
	\begin{enumerate}
	\item The inequality is tight.
	\item The inequality is not tight.
	\end{enumerate}
\end{problem}

\begin{problem}[Exercise 2.21]
What is the volume of the largest $d$-dimensional hypercube that can be placed entirely inside a unit radius $d$-dimensional ball? Argue that no larger cube can be placed.
\end{problem}

\begin{problem}[Exercise 2.47]
Let $x_1, x_2, \dots, x_n$ be independent samples of a random variable $\mathbf{x}$ with mean $\mu$ and variance $\sigma^2$. Let $m_s = \frac1n\sum_{i=1}^n x_i$ be the sample mean. Suppose one estimates the variance using the sample mean rather than the true mean, that is,
$$\sigma^2_s = \frac1n \sum_{i = 1}^n(x_i - m_s)^2.$$
Prove that $\mathbb E[\sigma^2_s]= \frac{n-1}{n} \sigma^2$ and thus one should have divided by $n-1$ rather than $n$.

\textbf{Hint}: First calculate the variance of the sample mean and show that $Var(m_s)=\frac1n Var(\mathbf{x})$.
Then calculate $\mathbb E[\sigma^2_s]  = \mathbb E[\frac1n \sum_{i = 1}^n (x_i - m_s)^2]$ by replacing $x_i - m_s$ with $(x_i - \mu) + (\mu - m_s)$.

\end{problem}


\begin{problem}[Exercise 2.49, Part 1]
Suppose you want to estimate the unknown center of a Gaussian in $d$-dimensional space which has variance one in each direction. 
Show that $O(\log d / \epsilon^2)$ random samples from the Gaussian are sufficient to get an estimate $\mathbf{m}$ of the true center $\mathbf{\mu}$, so that with probability at least $99/100$,
$$\max_i\left[|\mu_i - \textbf{m}_i|\right] \le \epsilon.$$
\end{problem}






\end{document}





