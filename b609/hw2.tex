%
% This is a borrowed LaTeX template file for lecture notes for CS267,
% Applications of Parallel Computing, UCBerkeley EECS Department and
%  CMU's 10725 Fall 2012 Optimization course
% taught by Geoff Gordon and Ryan Tibshirani.  
%Now it is used for CSCI-B609 Foundations of Data Science 
%course at Indiana  University. When preparing 
% your homework for this class, please use this template.
%
% To familiarize yourself with this template, the body contains
% some examples of its use.  Look them over.  Then you can
% run LaTeX on this file.  After you have LaTeXed this file then
% you can look over the result either by printing it out with
% dvips or using xdvi. "pdflatex template.tex" should also work.
%

\documentclass[twoside]{article}
\setlength{\oddsidemargin}{0.25 in}
\setlength{\evensidemargin}{-0.25 in}
\setlength{\topmargin}{-0.6 in}
\setlength{\textwidth}{6.5 in}
\setlength{\textheight}{8.5 in}
\setlength{\headsep}{0.75 in}
\setlength{\parindent}{0 in}
\setlength{\parskip}{0.1 in}

%
% ADD PACKAGES here:
%

\usepackage{amsmath,amsfonts,graphicx}

%
% The following commands set up the hwnum (homework number)
% counter and make various numbering schemes work relative
% to the homework number.
%
\newcounter{hwnum}
\renewcommand{\thepage}{\thehwnum-\arabic{page}}
\renewcommand{\thesection}{\thehwnum.\arabic{section}}
\renewcommand{\theequation}{\thehwnum.\arabic{equation}}
\renewcommand{\thefigure}{\thehwnum.\arabic{figure}}
\renewcommand{\thetable}{\thehwnum.\arabic{table}}

%
% The following macro is used to generate the header.
%
\newcommand{\hw}[4]{
   \pagestyle{myheadings}
   \thispagestyle{plain}
   \newpage
   \setcounter{hwnum}{#1}
   \setcounter{page}{1}
   \noindent
   \begin{center}
   \framebox{
      \vbox{\vspace{2mm}
    \hbox to 6.28in { {\bf CSCI B609 -- Foundations of Data Science
	\hfill Fall 2016} }
       \vspace{4mm}
       \hbox to 6.28in { {\Large \hfill Homework #1: #2  \hfill} }
       \vspace{2mm}
       \hbox to 6.28in { {\it Name: #3 \hfill #4} }
      \vspace{2mm}}
   }
   \end{center}


}
%
% Convention for citations is authors' initials followed by the year.
% For example, to cite a paper by Leighton and Maggs you would type
% \cite{LM89}, and to cite a paper by Strassen you would type \cite{S69}.
% (To avoid bibliography problems, for now we redefine the \cite command.)
% Also commands that create a suitable format for the reference list.
\renewcommand{\cite}[1]{[#1]}
\def\beginrefs{\begin{list}%
        {[\arabic{equation}]}{\usecounter{equation}
         \setlength{\leftmargin}{2.0truecm}\setlength{\labelsep}{0.4truecm}%
         \setlength{\labelwidth}{1.6truecm}}}
\def\endrefs{\end{list}}
\def\bibentry#1{\item[\hbox{[#1]}]}

%Use this command for a figure; it puts a figure in wherever you want it.
%usage: \fig{NUMBER}{SPACE-IN-INCHES}{CAPTION}
\newcommand{\fig}[3]{
			\vspace{#2}
			\begin{center}
			Figure \thehwnum.#1:~#3
			\end{center}
	}
% Use these for theorems, lemmas, proofs, etc.
\newtheorem{theorem}{Theorem}[hwnum]
\newtheorem{lemma}[theorem]{Lemma}
\newtheorem{proposition}[theorem]{Proposition}
\newtheorem{claim}[theorem]{Claim}
\newtheorem{corollary}[theorem]{Corollary}
\newtheorem{definition}[theorem]{Definition}
\newenvironment{proof}{{\bf Proof:}}{\hfill\rule{2mm}{2mm}}
\newtheorem{problem}[theorem]{Problem}


% **** IF YOU WANT TO DEFINE ADDITIONAL MACROS FOR YOURSELF, PUT THEM HERE:

\newcommand\E{\mathbb{E}}

\begin{document}

\hw{2}{October 04}{YOUR NAME HERE}{Due: Monday, October 17, 11:59pm EST}

\begin{problem}[Exercise 3.1, text modified]
Given a set of points $\{(x_i, y_i) | 1 \le i \le n\}$ give formulas for parameters $m$ and $b$ of the line of the form $y = mx + b$ that minimizes squared vertical distance between the points and the line (rather than the distance to the closest point on the line as we did in class). Formally if $(x_i, y_i)$ is a data point and $(x_i, y)$ is a point on the line then the vertical distance equals $|y_i - y|$.
\end{problem}

\begin{problem}[Problem 3.23, text modified]
\begin{enumerate}
\item For any matrix $A$ show that $\sigma_k \le \frac{\|A\|_F}{\sqrt{k}}$.
\item Prove that there exists a matrix $B$ of rank at most $k$ such that $\|A-B\|_2 \le \frac{\|A\|_F}{\sqrt{k}}$.
\item Does there exist a matrix $B$ of rank at most $k$ such that $\|A-B\|_F \le \frac{\|A\|_F}{\sqrt{k}}$? If yes, construct $B$, if no then give a counterexample.
\end{enumerate}
\end{problem}

\begin{problem}[Excercise 6.11]
What is the VC-dimension of the class $\mathcal H$ of axis-parallel boxes in $\mathbb R^d$. That is $\mathcal H = \{h_{a,b}: a,b \in \mathbb R^d\}$ where $h_{a,b}(x) = 1$ is $a_i \le x \le b_i$ for all $i = 1, \dots, d$ and $h_{a,b}(x) = 0$ otherwise.
\begin{enumerate}
\item Prove that the VC-dimension is at least your chosen $V$ by given a set of $V$ points that is shattered by the class (and explaining why it is shattered).
\item Prove taht the VC-dimension is at most your chosen $V$ by proving that no set of $V + 1$ points can be shattered.
\end{enumerate}

\end{problem}



\begin{problem}[Excercise 6.12]
Recall that the margin of a linear separator $w^*$ is defined as $\gamma = 1/\|w^*\|_2$.
Say that a set of points is shattered by linear separators of margin $\gamma$ if every laeling of the points in $S$ is achievable by a linear separator of margin at least $\gamma$. Prove that no set of $1/\gamma^2 + 1$ points in the unit ball is shattered by linear separators of margin $\gamma$.

\textbf{Hint:} think about the Perceptron algorithm and try a proof by contradiction.
\end{problem}



\end{document}





